\documentstyle{article}
\input{/usr/share/local/lib/tex/macroslocales/Dimensions.tex}

\title{PIPS: Dependence Graph}
\author{Franc,ois Irigoin \\
    Pierre Jouvelot \\
    Re'mi Triolet\\
    Yi-Qing Yang \\
\\
    CRI, Ecole des Mines de Paris}

\newcommand{\domain}[2]{\paragraph{{#1}}\paragraph{}{#2}}

\begin{document}
\maketitle

\sloppy

The dependence graph specific data structures, as well as auxiliary data
structures used to compute it are described in this report. The graph
itself is a generic graph defined in \verb/Newgen/graph.f.tex/. It is
specialized into a dependence graph by using specific arc and vertex
labels. 

The genericity is not supported explicitly by NewGen. Althoug the arc
and vertex labels are NewGen objects, they are seen as external
types. Specific procedures are specified in calls to
\verb/gen_init_external/. C module using these labels must use explicit
casts to convert the generic labels into dependence graph labels.

Auxiliary data structures are used to compute the strongly connected
components (SCC) of the dependence graph. The SCC routines are not as
generic as could be hoped. They take into account the graph hierarchy
implied by the dependence levels of the Allen \& Kennedy parallelization
algorithm. This explains why the SCC related data structures are
declared here and not with data structure \verb/graph/.

\iffalse
Les structures de donne'es suivantes sont utilise'es par la phase de
construction du graphe des de'pendances. Elles sont construites en
utilisant les structures de donne'es \verb+statement+ et \verb+effect+
qui ont e'te' de'finies dans la repre'sentation interne, ainsi que la
structure de donne'es ge'ne'riques \verb+vertex+ qui fait partie du
package {\em graph}.
\fi

\section*{Imported NewGen Types}

\iffalse
\domain{Import statement from "ri.newgen"}
{}
\fi

\domain{Import effect from "ri.newgen"}
{}

\domain{Import vertex from "graph.newgen"}
{}

\section*{Data Structures External to NewGen}

Generating systems are used to abstract a set of dependence arcs by
dependence cones (convex and transitive closure of the dependence set)
and/or dependence polyhedron (convex closure). This data structure is
part of the C3 Linear Algebra library.

\domain{External Ptsg}

\section*{Arc and Vertex Labels for the Dependence Graph}

\domain{dg\_vertex\_label = statement:int x sccflags}
{}

Ce domaine est utilise' pour contenir les informations qui sont
attache'es a` chaque noeud du graphe de de'pendances (voir le domaine
{\tt graph} dans le fichier {\tt graph.f.tex}). Le sous-domaine {\tt
statement} permet de retrouver l'instruction qui porte la de'pendance;
cet entier est le champ {\tt ordering} de l'instruction concerne'e. Le
sous-domaine {\tt sccflags} contient diverses informations utiles pour
le calcul des composantes fortement connexes.

\domain{dg\_arc\_label = conflicts:conflict*}
{}

Ce domaine est utilise' pour contenir les informations qui sont
attache'es a` chaque arc du graphe de de'pendances (voir le domaine {\tt
graph} dans le fichier {\tt graph.f.tex}). Chaque arc du GD contient les
conflits entre les deux statements des noeuds du graphe de de'pendance.

\domain{conflict = persistant source:effect x persistant sink:effect x cone}
{}

Un conflit existe entre deux effets pre'sents dans deux noeuds voisins
du graphe de de'pendance. Les types d'effet (write, read) sert a`
distinguer les conflits ``use-def'', ``def-def'' et ``{def-use}''. Le
cone comprend les informations precises de conflict.

\domain{cone = levels:int* x generating\_system:Ptsg}
{}

Le domaine co^ne de'finit une approximation polye'drique de l'ensemble
des de'pendances porte'es par un arc. Plusieurs types d'approximation
sont possibles: les niveaux de de'pendance, les vecteurs de direction de
de'pendance, les syste`mes ge'ne'rateurs (aussi
connus sous le nom de {\em co^ne de de'pendance}) et le me'canisme
du {\em Data Flow Graph} n'est pas imple'mente' avec cette structure
de donne'es mais avec \verb|paf_ri|.

Le niveau d'une de'pendance d\'ecrivent le nombre de boucles englobantes concerne\'ees.
``{generating\_system}'' est la r\'epresentation de system generateur.

\section*{Strongly Connected Components}

\domain{Sccflags = enclosing\_scc:scc x mark:int x dfnumber:int x lowlink:int}
{}

Ce domaine est utilise' par l'algorithme de calcul des composantes
fortement connexes d'un graphe.

\domain{Sccs = sccs:scc*}
{}

Ce domaine permet de contenir le re'sultat de l'algorithme de calcul
des composantes fortement connexes d'un graphe. Il s'agit d'une liste de
{\tt scc}, c'est a` dire de composantes fortement connexes.

\domain{Scc = vertices:vertex* x indegree:int}
{}

Ce domaine permet de repre'senter une composante fortement connexe d'un
graphe. Une {\tt scc} se compose d'un ensemble de noeuds, et d'un
sous-domaine {\tt indegree} utilise' par l'algorithme de tri topologique.

\end{document}
\end
