\documentclass[a4paper]{article}

\usepackage[latin1]{inputenc}
\usepackage{newgen_domain}
\usepackage{listings}
\usepackage[backref,pagebackref]{hyperref}

\title{Cloning statement}
\author{Serge Guelton}


\begin{document}
\maketitle
\sloppy

\section*{New type used for cloning}

This document defines and describes the data structure used to pass parameters to the \lstinline|clone_statement| function.
It consist of three fields given by the user:
\begin{itemize}
\item \lstinline|previous_module| is the entity of the module from which the cloned statement comes
\item \lstinline|new_module| is the entity of the module to which the cloned statement will be added. It can be the same as \lstinline|previous_module|
\item \lstinline|new_module_statement| is the root statement of \lstinline|new_module|
\end{itemize}

\domain{import entity from "ri.newgen"}
{}
\domain{import statement from "ri.newgen"}
{}



\domain{clone_context = previous_module:entity x new_module:entity x new_module_statement:statement}

\end{document}
