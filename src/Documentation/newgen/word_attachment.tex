\documentstyle{article}

\input{/usr/share/local/lib/tex/macroslocales/Dimensions.tex}

\title{Attachment of some information to text words\\Data structures}
\author{Ronan Keryell\\
                \\
        CRI, E'cole des mines de Paris}

\newcommand{\domain}[2]{\paragraph{{#1}}\paragraph{}{{#2}}}

\begin{document}

\section*{Introduction}

This report present the data structures used to attach some
information to some word of text, mainly to use in the hypertext Emacs
interface.

\domain{import entity from "ri.newgen"}
\domain{import expression from "ri.newgen"}

\section{Attachments}

Are only a list of {\tt attachment}s:

\domain{attachments = persistent attachment*}

{\tt attachment} is persistent because an attachment is referenced
twice.

\section{Attachment}

Here is what can be attached to a word:

\domain{attachment = attachee x begin:int x end:int}

{\tt begin} is the position of the attachment begin in the output
file and {\tt end} the position of its end.

The various objects that can be attached:
\domain{attachee = persistent entity + unknown:unit}

Persistance is need because we do not want the RI to be broken when
the attachments are freed.

\section{The mapping used to attach various internal informations}

%Each attachment point to a word of the code:
%\domain{attachment_to_word = attachment->string}

Each word can have a list of attachments:
\domain{word_to_attachments = persistent string->attachments}

The {\tt string} is persistent since it is already freed by the
prettyprinter (in fact in \verb|print_sentence()|.


\end{document}
