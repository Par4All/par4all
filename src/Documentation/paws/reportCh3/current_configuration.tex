\section{Current configuration}
\label{currentconfiguration}

This section contains information necessary to understand how PaWS is configured.

\subsection{Configuration file}

The initial settings for PaWS are stored in a configuration file named \emph{paws.py} in Pylons configuration directory \emph{pawsapp/config}. Parameters stored there are i.e. paths to the directories with PaWS helper files.

The configuration file is written in Python. Thanks to that it can be imported as a Python module and no parsing of the file is needed to extract settings. Parameters can also be defined as a Python variables.

\subsection{Introduction page}

The entry web page (\emph{pass/index}) contains information about all the PaWS modes and their cases. The site is hosted by template \emph{paas.mako} and controller \emph{paas.py}, but the content of the site is based on the structure of description files described below. During each site loading, information about available tools and their descriptions is loaded from that directory. The site HTML code is generated dynamically from scratch using this information.

When a new tool is added, a script added creates the description file and places it in the right place of the structure.

\subsection{Descriptions files}

Description files are placed in validation \emph{validation/Paws} directory. This directory consists of subdirectories corresponding to the particular PaWS modes and two extra description files located in \emph{validation/Paws/main} directory: \emph{paas.txt} and \emph{functionalities.txt}. 

The subdirectories contain the description files for all of the cases of the mode they are refering to. For example, the subdirectory of the {\bf tools} mode currently contains files: \emph{in\_regions.txt}, \emph{openmp.txt}, \emph{out\_regions.txt}, \emph{preconditions.txt} and \emph{regions.txt}.

The first file from \emph{main} directory, \emph{paas.txt}, provides a general description of the PaWS framework. This description is visible on the main entrance page. The second file, \emph{functionalities.txt}, gathers information about PaWS modes: their title, visible on the entry page, name of the subdirectory with case descriptions, and flag indicating whether mode has two levels, basic and advanced, or not.

\subsection{Analysis and transformations}

Currently, there are four analyses and one transformation available: \emph{preconditions}, \emph{OpenMP}, \emph{IN regions}, \emph{OUT regions} and \emph{regions}. Each of them has basic level. Two of them, \emph{preconditions} and \emph{regions} also have an advanced level configured.

As written above, descriptions of the analysis and transformation tools are placed in the \emph{validation/Paws/tools} subdirectories and examples are in the \emph{validation/Paws/$<$tool\_name$>$} directories.

The \emph{skeleton.mako} template is responsible for code of all the sites generation. There are two more specific templates inheriting from it - for basic level \emph{base.mako} and for advanced one \emph{advanced\_base.mako}. Together they provide complete support for all analyses and transformations.

Pages dedicated to a different tool, both for basic and advanced mode, are hosted by a special controller, named \emph{tools\_$<$tool\_name$>$.py} and are customized by a Mako template, named \emph{tools\_$<$tool\_name$>$.mako} or \emph{tools\_$<$tool\_name$>$\_advanced.mako} for advanced mode. Controller and templates are created automatically when a new tool is added as described in Section \ref{add_analysis_transformation}. For detailed files structure see Figure \ref{fig:paws_structure}.

\subsubsection{Advanced mode}

The advanced mode of analysis and transformations adds more user control to the base level. It is possible to choose properties which will be used when performing operation. The user is also able to select concrete kind of analysis (or transformation) that will be activated and phases to be applied before the result is displayed. 

Advanced mode examples are stored in \emph{validation/Paws/tools/$<$tool\_name$>$} directories and they are shared with the basic mode.

Content of the web pages forms for those settings selection is created dynamically. It is based on the lists of possible properties, analyses and phases. Those lists are stored in \emph{validation/Paws/tools/$<$tool\_name$>$} directories. Each of those files has its own structure:

\begin{itemize}

  \item {\bf Properties} are grouped in three categories: boolean, integer and string properties. Each property should be declared in the new line as it is shown in the Listing \ref{PropertyDeclaration}\footnote{Sign \emph{``-''} means that any value is possible.}.
  
\lstset{language=Python,caption={Property declaration},label=PropertyDeclaration}
\begin{lstlisting}
PROPERTY_NAME;default_value;value_1;other_values...;description
\end{lstlisting}

  \item {\bf Analyses} are grouped by modules. In example module \emph{preconditions} has 4 analyses (\emph{preconditions\_intra\_full}, \emph{preconditions\_inter\_fast}, \emph{preconditions\_intra\_fast}, \emph{preconditions\_intra}). In a given module, only one analysis can be selected. Analyses should be declared in the new line as it is in the Listing \ref{AnalysesDeclaration}.
  
  \lstset{language=Python,caption={Analyses declaration},label=AnalysesDeclaration}
\begin{lstlisting}
<<MODULE_NAME>>
default_analysis_name;description
analysis_1_name;description
...
\end{lstlisting}
  
  \item {\bf Phases} are also declared in new lines each. Example is shown in the Listing \ref{PhasesDeclaration}.
  
  \lstset{language=Python,caption={Phases declaration},label=PhasesDeclaration}
\begin{lstlisting}
phases_1_name;default_value;description
phases_2_name;default_value;description
...
\end{lstlisting}

\end{itemize}

\subsection{Demonstrations}

There are three demonstrations available: \emph{aile\_excerpt}, \emph{convol} and \emph{acca-2011}. Descriptions of all the cases are placed in \emph{pawsapp/public/descriptions/tutorial} directory.

Each demonstration consists in an example source code, in C or Fortran, and a Tpips script. They are found in \emph{validation/Paws/demo} directory.

Controller \emph{demo.py} and template \emph{demo.mako} are providing skeleton of the HTML code and basic functions common to all the demonstrations. As in the case of analysis and transformations, each demonstration is hosted by specific controller (named \emph{tutorial\_$<$demo\_name$>$.py}) and a Mako template (named \emph{tutorial\_$<$demo\_name$>$.mako}). The automatic creation of them is described in Section \ref{add_demonstration}.

Each Tpips script used in a demonstration is parsed and divided in several steps. Markers for separating the steps are the lines starting with the command ``display''. It is the signal that the result of the operations should be shown to the user. Not only result of the step is displayed, but also the Tpips commands of the step are presented to the user.

A second thing which is very important for those who are designing demonstration cases, is the ways comments can be inserted in the output. There are two ways to do that:

\begin{itemize}
  \item {\bf comments in Tpips script output} are simple bash comments as it is shown in the Listing \ref{TpipsComments}:
  
  \lstset{language=Python,caption={Comments in Tpips script output},label=TpipsComments}
  \begin{lstlisting}
# Remove existing workspace, if any
delete acca-2011
  \end{lstlisting}
  
  They are displayed as a part of the Tpips script linked to the current step.
  
  \item {\bf comments in result output} are created by writing message to standard output with \emph{echo}\cite{echo} bash\cite{bash} command. It is important to remember that comments should be suitable with the relevant programming language comments style. Example of creating comments for C output is shown in Listing \ref{EchoComments}:
    
  \lstset{language=bash,caption={Comments in step result output},label=EchoComments}
  \begin{lstlisting}
echo '/*'
echo ' * Source code for "compute" function.'
echo ' */'
echo
  \end{lstlisting}
  
  These comments are displayed in the output window.
  
\end{itemize}

\subsection{Examples}

All examples for all of analysis and transformations are stored in the validation directory \emph{validation/Paws}. Adding new example is very easy and the process is described in the Section \ref{add_example}.

There is special controller \emph{examples.py} designed to handle the examples. It is responsible for getting information about available examples and for loading file contents.

\subsection{Python code}

Python code is located in the \emph{validation/Paws/pyps\_modules} directory. File \emph{paws\_base.py} is responsible for the basic operation, such as opening and closing the workspace or setting properties. Each analysis or transformation has its own file with \emph{invoke\_function} method which is responsible for concreate operation, i.e. for preconditions, file \emph{paws\_preconditions.py} contains code for getting preconditions code.