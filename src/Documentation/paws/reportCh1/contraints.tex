\section{Design constraints}
\label{design_contraints}

As it is written in the previous section, main goal of the PaWS framework is to create a very light WEB interface for PIPS. It also subjects to other, more detailed constraints:

\begin{enumerate}
  \item {\bf No installation requirements} on a the client side, beyond the browser.
  \item {\bf PIPS scripting complexity is hidden}, result of the operation is ready in few steps.
  \item {\bf Separation of PIPS, PyPS and PaWS} to avoid dependencies between pieces of software and to protect web server from crushing caused by the internal errors.
  \item {\bf Easy customization} of the application for users and developers who are not familiar with PaWS and Pylons. Process of changing particular element is easy, because application is based on the structure of the files which contains python modules with pyps functionalities, examples and files with descriptions of them.
  \item {\bf PIPS server is updated} with recent version of PIPS, wich is propely validated.
  \item {\bf Security} against spamming engines is obtained through the inability to provide Python or shell input.
  \item {\bf Consistency} between:
    \begin{itemize}
      \item the output computed by Tpips and by Pyps;
      \item the output of the cut-and-pasted code, code loaded from the PaWS example, code loaded from the user machine and modified code.
    \end{itemize}
  \item {\bf Using of PIPS validation mechanism}
  \item {\bf Protection against server overload} to assure PaWS availability; control response time.
  \item {\bf Reusability} of existing components as much as possible and generic approach to page creation.
  \item {\bf OS neutrality}
\end{enumerate}

%%  \item {\bf remote execution of PIPS -> demo? not sure}