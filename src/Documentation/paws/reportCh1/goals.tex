\section{PaWS Project}
\label{paws_project}

PaWS provides three different modes of using PIPS: tutorial, elementary analysis or transformation and full control. Each of them presents different approach to try how PIPS is working, according to the level of users advance.

%% see Introduction for goals
%% implementation vs configuration

\subsection{Tutorial mode}
Tutorial is PaWS mode which presents PIPS pass managers to users, who are not familiar with this framework. It is also very easy way to learn how PIPS is working. The user needs only to choose an example and after it is loaded, he/she can follow transformations and analysis step-by-step. The user can also skip some steps or go back to previous ones. There is always a script and its results are presented with some explanations. The result may include dependence graphs for pedagogical reasons.

\subsection{Basic analyses and transformations}
This mode enables intermediate users to try specific PIPS transformations and analysis. The user can choose prepared examples or use his own code to check how PIPS works. He can later modify the source code to see differences in results. PIPS related analyses and transformations are available in two levels: basic and advanced. Basic level performs operations with default PIPS properties. Advanced level provides list of properties to chosen tool, which can be modified by the user. More information can be found in configuration chapter (\ref{currentconfiguration}).

\subsection{Full control}
Full control is the mode which enable user to create graphically PyPS or TPIPS scripts which are applied later over the source code.

HOW IT IS DONE - to be continued.