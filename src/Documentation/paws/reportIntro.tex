PaWS is a web interface for the PIPS framework. PIPS is the source-to-source compilation framework (for more detailed information check section \ref{pips_and_pyps}) created by Centre de recherche en informatique of MINES ParisTech \cite{cri}. It is a very complex framework and process its installation is complicated because of many dependencies. With PaWS people can try PIPS without installing it.

With PaWS, user can learn how PIPS works. There are prepared examples which demonstrate PIPS abilities, but users can also try PIPS analyse and transformations on his/her own source code and finally try to change the way how PIPS is working. PaWS provides different levels for different stages of users advance.

PaWS, as it is a web interface, it is a light tool to use, 'thin client', that require no installation but only a bootstraper. Graphical interface is also more intuitive and easy to use for user.

There is also possibility of downloading and setting up user own PaWS instance.

In this document you find a detailed description of PaWS design constraints, design choices, report of the process of creating framework and problems encountered. There is also a technical documentation of the implementation and the configuration of PaWS in section \ref{configuration_guide}. The last section of this document is a user guide.
%% provides information about usage of PaWS.