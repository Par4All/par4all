\section{Basic tools}

Basic tool provides PIPS analysis or transformation operations. They are available in two levels: base and advanced (see Section \ref{paws_project}).

\subsection{Basic level}

Basic level allows the user to try PIPS analysis or transformation with a default settings. Example page is shown in the Figure \ref{fig:basic_mode_screen}.

\begin{figure}[h!]
  \centering
  \includegraphics[width=0.8\textwidth]{reportCh4/basic_mode_screen}
  \caption{Preconditions basic level screen.}
  \label{fig:basic_mode_screen}
\end{figure}

User can provide the source code either by loading one of supplied examples ({\bf 1.}) or by picking it form his/her machine ({\bf 2.}) or by typing it in the source code window ({\bf 3.}). Operation is performed by clicking either the button ``RUN'' ({\bf 4.}) or tab ``PRECONDITIONS'' ({\bf 5.}). Selecting the tab ``GRAPH'' ({\bf 6.}) invokes dependence graph creation.

After loading the source code, language of the code is detected and displayed in the programming language label ({\bf 7.}). In case of inconvenience, the user can change font size used on the side by the ``A+'' and ``A-'' buttons ({\bf 8.}).

The user can save or print the results of the performed operation. Buttons, which enable that ({\bf 9.} and {\bf 10.}), disactivated at the beginning, are activated when the result is ready.

Link at the bottom of the page ({\bf 11.}) leads to the advanced level of the same tool.

\subsection{Advanced level}

The layout and usage of the advanced level page is very similar to the basic one. The only difference if possibility of setting properties, analyses and phases (see the figure \ref{fig:advanced_properties}). Each setting has its description which is visible after hovering the mouse over its check button.

Boolean properties ({\bf 1.}) are set to {\bf True} or {\bf False} by simple being checked. Integer properties ({\bf 2.}) are checked by a default and their values can be changed by the user. After invocation of the operation they are validated. String properties ({\bf 3.}), analyses ({\bf 4.}) and phases ({\bf 5.}) have got prepared list of possible values. Default one is already set.

\begin{figure}[h!]
  \centering
  \includegraphics[width=0.5\textwidth]{reportCh4/advanced_properties}
  \caption{Forms for setting properties, analyses and phases.}
  \label{fig:advanced_properties}
\end{figure}