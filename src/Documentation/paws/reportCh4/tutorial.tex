\section{Tutorial}

Aim of the tutorial mode is to demonstrate the user variety of PIPS abilities. At the begining, chosen example is loaded, as it is shown in the Figure \ref{fig:tutorial_screen} (example ``Acca-2011''). 

\begin{figure}[h!]
  \centering
  \includegraphics[width=0.8\textwidth]{reportCh4/tutorial_screen}
  \caption{``Acca-2011'' tutorial screen.}
  \label{fig:tutorial_screen}
\end{figure}

This mode does not require significant activity of the user - the source code ({\bf 1.}) and the script with operations and comments ({\bf 2.}) are provided. The user only needs to change current step using the slider ({\bf 3.}). It is very flexible, stepping does not have to be sequential - the user can go back or skip some steps. Current step and the number of all steps are displayed over the slider ({\bf 4.}).